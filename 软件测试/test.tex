\documentclass[UTF8,a4paper,10pt]{ctexart}
\usepackage{graphicx}

\usepackage[left=2.50cm, right=2.50cm, top=2.50cm, bottom=2.50cm]{geometry} %页边距
\CTEXsetup[format={\Large\bfseries}]{section} %设置章标题居左
 
 
%%%%%%%%%%%%%%%%%%%%%%%
% -- text font --
% compile using Xelatex
%%%%%%%%%%%%%%%%%%%%%%%
% -- 中文字体 --
%\setmainfont{Microsoft YaHei}  % 微软雅黑
%\setmainfont{YouYuan}  % 幼圆    
%\setmainfont{NSimSun}  % 新宋体
%\setmainfont{KaiTi}    % 楷体
%\setmainfont{SimSun}   % 宋体
%\setmainfont{SimHei}   % 黑体
% -- 英文字体 --
%\usepackage{times}
%\usepackage{mathpazo}
%\usepackage{fourier}
%\usepackage{charter}
\usepackage{helvet}
 
 
\usepackage{amsmath, amsfonts, amssymb} % math equations, symbols
\usepackage[english]{babel}
\usepackage{color}      % color content
\usepackage{graphicx}   % import figures
\usepackage{url}        % hyperlinks
\usepackage{bm}         % bold type for equations
\usepackage{multirow}
\usepackage{booktabs}
\usepackage{epstopdf}
\usepackage{epsfig}
\usepackage{algorithm}
\usepackage{algorithmic}
\renewcommand{\algorithmicrequire}{ \textbf{Input:}}     % use Input in the format of Algorithm  
\renewcommand{\algorithmicensure}{ \textbf{Initialize:}} % use Initialize in the format of Algorithm  
\renewcommand{\algorithmicreturn}{ \textbf{Output:}}     % use Output in the format of Algorithm  
 
 
\usepackage{fancyhdr} %设置页眉、页脚
%\pagestyle{fancy}
\lhead{}
\chead{}
%\rhead{\includegraphics[width=1.2cm]{fig/ZJU_BLUE.eps}}
\lfoot{}
\cfoot{}
\rfoot{}
 
 
%%%%%%%%%%%%%%%%%%%%%%%
%  设置水印
%%%%%%%%%%%%%%%%%%%%%%%
%\usepackage{draftwatermark}         % 所有页加水印
%\usepackage[firstpage]{draftwatermark} % 只有第一页加水印
% \SetWatermarkText{Water-Mark}           % 设置水印内容
% \SetWatermarkText{\includegraphics{fig/ZJDX-WaterMark.eps}}         % 设置水印logo
% \SetWatermarkLightness{0.9}             % 设置水印透明度 0-1
% \SetWatermarkScale{1}                   % 设置水印大小 0-1    
 
\usepackage{hyperref} %bookmarks

\hypersetup{
colorlinks=true,
linkcolor=black
}
\title{\textbf{对软件本质的认识与信息的作用}}
\author{ 阮海航}
 
\begin{document}
    \maketitle

\section{对软件的认识}
软件是一种固化的思维,这一点决定了许多的事情。从特质上来看,既然软件是固化的思维,那就必然同时具备思维以及思维所承载之物之特质。既然思维自身的特质是复合的,那么作为固化思维的软件,其特质必然也是复合的:这也就意味着在软件这一大的范畴里,两种矛盾的说法同时成立,并不是什么值得惊讶的事情。只要思维承载的东西蕴含着彼此对立的东西,那么两种对立的观点同属于软件这一范畴,并且同时争取,一点也不稀奇。这很可能是大家吵来吵去的一个根本原因,因为我们总是喜欢用自己的经历来定义软件是什么以及判断标准,但如果这种经历来自完全不同的两个领域,并且互相矛盾,那就只能吵架。实际上只有基于软件是一种思维这样特质推导出来的东西才更有普适性。软件就是按既定要求进行的运算、储存、读取、传输。既定要求的输入包括键盘输入、端口输入、网络传输,其他方式输入。运算结果的输出包括屏幕显示、打印输出、网络传输,其他方式输出。

\section{信息的作用}
\subsection{对于人类的作用}
信息作为一种客观存在,它一直都在积极地发挥着人类意识或没有意识到的重大作用。科学技术在近两个世纪所取得的空前进步,使人们终于认识到,信息是与物质和能源可以相提并论的用以维系人类社会存在及发展的三大要素之一。信息给人类社会带来了多元化的变化,吃喝住行等等方面,丰富了人民的生活。信息是非常重要的,信息首先重要在它对决策起作用,能够集中精力在更能够出业绩、出成效的方向,而不是走错路、浪费时间、白费力气。信息其次重要在它让我们更完全,不盲信。哪怕你明知真实的信息却利用它去欺骗了别人(利用信息不对称获取利益),你也不会是在这种相对关系中,那个更傻的人。然而,我们是不是都需要让自己不至于去落入那种傻欺的境地。
\subsection{对于国家的作用}
同时,信息对于国家的发展具有重大的意义,在工业发达国家,信息经济正迅速发展成为指导现代经济的主要经济,并且对世界各国的经济发展都产生了重大的影响和推进。近些年来,我国信息产业的发展异常迅速,信息经济产值的快速增长已很好地证明了信息在经济发展中所起的巨大作用。
\section{我的个人感悟}
信息首先重要在它对决策起作用,能够集中精力在更能够出业绩、出成效的方向,而不是走错路、浪费时间、白费力气。

信息其次重要在它让我们更完全,不盲信。\par
因为通过学习和摸索做到这样,在跟不同行业以及相关行业的人士交流过程中,我了解到一些极为有用的信息,这种信息于我又在别处进一步得到了验证,这些有用的信息配合相关的工作经历,可能,会对我将来的工作产生很大的影响和帮助。\par

而这些信息,不管是由经历换来,还是由交流换来,都是我愈加看重的珍贵“财富”的一部分。\par
\bibliographystyle{plain}
\bibliography{sample}
\end{document}